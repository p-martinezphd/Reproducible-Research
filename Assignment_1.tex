% Options for packages loaded elsewhere
\PassOptionsToPackage{unicode}{hyperref}
\PassOptionsToPackage{hyphens}{url}
%
\documentclass[
]{article}
\usepackage{amsmath,amssymb}
\usepackage{lmodern}
\usepackage{iftex}
\ifPDFTeX
  \usepackage[T1]{fontenc}
  \usepackage[utf8]{inputenc}
  \usepackage{textcomp} % provide euro and other symbols
\else % if luatex or xetex
  \usepackage{unicode-math}
  \defaultfontfeatures{Scale=MatchLowercase}
  \defaultfontfeatures[\rmfamily]{Ligatures=TeX,Scale=1}
\fi
% Use upquote if available, for straight quotes in verbatim environments
\IfFileExists{upquote.sty}{\usepackage{upquote}}{}
\IfFileExists{microtype.sty}{% use microtype if available
  \usepackage[]{microtype}
  \UseMicrotypeSet[protrusion]{basicmath} % disable protrusion for tt fonts
}{}
\makeatletter
\@ifundefined{KOMAClassName}{% if non-KOMA class
  \IfFileExists{parskip.sty}{%
    \usepackage{parskip}
  }{% else
    \setlength{\parindent}{0pt}
    \setlength{\parskip}{6pt plus 2pt minus 1pt}}
}{% if KOMA class
  \KOMAoptions{parskip=half}}
\makeatother
\usepackage{xcolor}
\IfFileExists{xurl.sty}{\usepackage{xurl}}{} % add URL line breaks if available
\IfFileExists{bookmark.sty}{\usepackage{bookmark}}{\usepackage{hyperref}}
\hypersetup{
  pdftitle={Reproducible\_Research\_Assingment\_1},
  pdfauthor={Paul M.},
  hidelinks,
  pdfcreator={LaTeX via pandoc}}
\urlstyle{same} % disable monospaced font for URLs
\usepackage[margin=1in]{geometry}
\usepackage{color}
\usepackage{fancyvrb}
\newcommand{\VerbBar}{|}
\newcommand{\VERB}{\Verb[commandchars=\\\{\}]}
\DefineVerbatimEnvironment{Highlighting}{Verbatim}{commandchars=\\\{\}}
% Add ',fontsize=\small' for more characters per line
\usepackage{framed}
\definecolor{shadecolor}{RGB}{248,248,248}
\newenvironment{Shaded}{\begin{snugshade}}{\end{snugshade}}
\newcommand{\AlertTok}[1]{\textcolor[rgb]{0.94,0.16,0.16}{#1}}
\newcommand{\AnnotationTok}[1]{\textcolor[rgb]{0.56,0.35,0.01}{\textbf{\textit{#1}}}}
\newcommand{\AttributeTok}[1]{\textcolor[rgb]{0.77,0.63,0.00}{#1}}
\newcommand{\BaseNTok}[1]{\textcolor[rgb]{0.00,0.00,0.81}{#1}}
\newcommand{\BuiltInTok}[1]{#1}
\newcommand{\CharTok}[1]{\textcolor[rgb]{0.31,0.60,0.02}{#1}}
\newcommand{\CommentTok}[1]{\textcolor[rgb]{0.56,0.35,0.01}{\textit{#1}}}
\newcommand{\CommentVarTok}[1]{\textcolor[rgb]{0.56,0.35,0.01}{\textbf{\textit{#1}}}}
\newcommand{\ConstantTok}[1]{\textcolor[rgb]{0.00,0.00,0.00}{#1}}
\newcommand{\ControlFlowTok}[1]{\textcolor[rgb]{0.13,0.29,0.53}{\textbf{#1}}}
\newcommand{\DataTypeTok}[1]{\textcolor[rgb]{0.13,0.29,0.53}{#1}}
\newcommand{\DecValTok}[1]{\textcolor[rgb]{0.00,0.00,0.81}{#1}}
\newcommand{\DocumentationTok}[1]{\textcolor[rgb]{0.56,0.35,0.01}{\textbf{\textit{#1}}}}
\newcommand{\ErrorTok}[1]{\textcolor[rgb]{0.64,0.00,0.00}{\textbf{#1}}}
\newcommand{\ExtensionTok}[1]{#1}
\newcommand{\FloatTok}[1]{\textcolor[rgb]{0.00,0.00,0.81}{#1}}
\newcommand{\FunctionTok}[1]{\textcolor[rgb]{0.00,0.00,0.00}{#1}}
\newcommand{\ImportTok}[1]{#1}
\newcommand{\InformationTok}[1]{\textcolor[rgb]{0.56,0.35,0.01}{\textbf{\textit{#1}}}}
\newcommand{\KeywordTok}[1]{\textcolor[rgb]{0.13,0.29,0.53}{\textbf{#1}}}
\newcommand{\NormalTok}[1]{#1}
\newcommand{\OperatorTok}[1]{\textcolor[rgb]{0.81,0.36,0.00}{\textbf{#1}}}
\newcommand{\OtherTok}[1]{\textcolor[rgb]{0.56,0.35,0.01}{#1}}
\newcommand{\PreprocessorTok}[1]{\textcolor[rgb]{0.56,0.35,0.01}{\textit{#1}}}
\newcommand{\RegionMarkerTok}[1]{#1}
\newcommand{\SpecialCharTok}[1]{\textcolor[rgb]{0.00,0.00,0.00}{#1}}
\newcommand{\SpecialStringTok}[1]{\textcolor[rgb]{0.31,0.60,0.02}{#1}}
\newcommand{\StringTok}[1]{\textcolor[rgb]{0.31,0.60,0.02}{#1}}
\newcommand{\VariableTok}[1]{\textcolor[rgb]{0.00,0.00,0.00}{#1}}
\newcommand{\VerbatimStringTok}[1]{\textcolor[rgb]{0.31,0.60,0.02}{#1}}
\newcommand{\WarningTok}[1]{\textcolor[rgb]{0.56,0.35,0.01}{\textbf{\textit{#1}}}}
\usepackage{graphicx}
\makeatletter
\def\maxwidth{\ifdim\Gin@nat@width>\linewidth\linewidth\else\Gin@nat@width\fi}
\def\maxheight{\ifdim\Gin@nat@height>\textheight\textheight\else\Gin@nat@height\fi}
\makeatother
% Scale images if necessary, so that they will not overflow the page
% margins by default, and it is still possible to overwrite the defaults
% using explicit options in \includegraphics[width, height, ...]{}
\setkeys{Gin}{width=\maxwidth,height=\maxheight,keepaspectratio}
% Set default figure placement to htbp
\makeatletter
\def\fps@figure{htbp}
\makeatother
\setlength{\emergencystretch}{3em} % prevent overfull lines
\providecommand{\tightlist}{%
  \setlength{\itemsep}{0pt}\setlength{\parskip}{0pt}}
\setcounter{secnumdepth}{-\maxdimen} % remove section numbering
\ifLuaTeX
  \usepackage{selnolig}  % disable illegal ligatures
\fi

\title{Reproducible\_Research\_Assingment\_1}
\author{Paul M.}
\date{2022-10-31}

\begin{document}
\maketitle

\hypertarget{loading-and-preprocessing-the-data}{%
\subsection{Loading and preprocessing the
data}\label{loading-and-preprocessing-the-data}}

\begin{Shaded}
\begin{Highlighting}[]
\NormalTok{data }\OtherTok{\textless{}{-}} \FunctionTok{read.csv}\NormalTok{(}\StringTok{\textquotesingle{}\textasciitilde{}/activity.csv\textquotesingle{}}\NormalTok{)}
\end{Highlighting}
\end{Shaded}

\#\#What is mean total number of steps taken per day? \#For this part of
the assignment, you can ignore the missing values in the dataset.

\begin{Shaded}
\begin{Highlighting}[]
\NormalTok{daysteps }\OtherTok{\textless{}{-}} \FunctionTok{tapply}\NormalTok{(data}\SpecialCharTok{$}\NormalTok{steps, data}\SpecialCharTok{$}\NormalTok{date, sum, }\AttributeTok{na.rm=}\ConstantTok{TRUE}\NormalTok{)}
\FunctionTok{summary}\NormalTok{(daysteps)}
\end{Highlighting}
\end{Shaded}

\begin{verbatim}
##    Min. 1st Qu.  Median    Mean 3rd Qu.    Max. 
##       0    6778   10395    9354   12811   21194
\end{verbatim}

\hypertarget{make-a-histogram-of-the-total-number-of-steps-taken-each-day}{%
\paragraph{1. Make a histogram of the total number of steps taken each
day}\label{make-a-histogram-of-the-total-number-of-steps-taken-each-day}}

\begin{Shaded}
\begin{Highlighting}[]
\FunctionTok{qplot}\NormalTok{(daysteps, }\AttributeTok{xlab=}\StringTok{\textquotesingle{}Steps Per Day\textquotesingle{}}\NormalTok{, }\AttributeTok{ylab=}\StringTok{\textquotesingle{}Frequency\textquotesingle{}}\NormalTok{, }\AttributeTok{binwidth=}\DecValTok{500}\NormalTok{)}
\end{Highlighting}
\end{Shaded}

\includegraphics{Assignment_1_files/figure-latex/unnamed-chunk-4-1.pdf}

\hypertarget{calculate-and-report-the-mean-and-median-total-number-of-steps-taken-per-day}{%
\paragraph{2. Calculate and report the mean and median total number of
steps taken per
day}\label{calculate-and-report-the-mean-and-median-total-number-of-steps-taken-per-day}}

\begin{Shaded}
\begin{Highlighting}[]
\NormalTok{daystepsMean }\OtherTok{\textless{}{-}} \FunctionTok{mean}\NormalTok{(daysteps)}
\NormalTok{daystepsMedian }\OtherTok{\textless{}{-}} \FunctionTok{median}\NormalTok{(daysteps)}
\end{Highlighting}
\end{Shaded}

\begin{itemize}
\tightlist
\item
  Mean: 9354.2295082
\item
  Median: 10395
\end{itemize}

\hypertarget{what-is-the-average-daily-activity-pattern}{%
\subsection{What is the average daily activity
pattern?}\label{what-is-the-average-daily-activity-pattern}}

\begin{Shaded}
\begin{Highlighting}[]
\NormalTok{averagep}\OtherTok{\textless{}{-}} \FunctionTok{aggregate}\NormalTok{(}\AttributeTok{x=}\FunctionTok{list}\NormalTok{(}\AttributeTok{meanSteps=}\NormalTok{data}\SpecialCharTok{$}\NormalTok{steps), }\AttributeTok{by=}\FunctionTok{list}\NormalTok{(}\AttributeTok{interval=}\NormalTok{data}\SpecialCharTok{$}\NormalTok{interval), }\AttributeTok{FUN=}\NormalTok{mean, }\AttributeTok{na.rm=}\ConstantTok{TRUE}\NormalTok{)}
\FunctionTok{summary}\NormalTok{(averagep)}
\end{Highlighting}
\end{Shaded}

\begin{verbatim}
##     interval        meanSteps      
##  Min.   :   0.0   Min.   :  0.000  
##  1st Qu.: 588.8   1st Qu.:  2.486  
##  Median :1177.5   Median : 34.113  
##  Mean   :1177.5   Mean   : 37.383  
##  3rd Qu.:1766.2   3rd Qu.: 52.835  
##  Max.   :2355.0   Max.   :206.170
\end{verbatim}

\hypertarget{make-a-time-series-plot}{%
\paragraph{1. Make a time series plot}\label{make-a-time-series-plot}}

\begin{Shaded}
\begin{Highlighting}[]
\FunctionTok{ggplot}\NormalTok{(}\AttributeTok{data=}\NormalTok{averagep, }\FunctionTok{aes}\NormalTok{(}\AttributeTok{x=}\NormalTok{interval, }\AttributeTok{y=}\NormalTok{meanSteps)) }\SpecialCharTok{+}
    \FunctionTok{geom\_line}\NormalTok{() }\SpecialCharTok{+}
    \FunctionTok{xlab}\NormalTok{(}\StringTok{"5 minute interval"}\NormalTok{) }\SpecialCharTok{+}
    \FunctionTok{ylab}\NormalTok{(}\StringTok{"average steps"}\NormalTok{) }
\end{Highlighting}
\end{Shaded}

\includegraphics{Assignment_1_files/figure-latex/unnamed-chunk-7-1.pdf}

\hypertarget{which-5-minute-interval-on-average-across-all-the-days-in-the-dataset-contains-the-maximum-number-of-steps}{%
\paragraph{2. Which 5-minute interval, on average across all the days in
the dataset, contains the maximum number of
steps?}\label{which-5-minute-interval-on-average-across-all-the-days-in-the-dataset-contains-the-maximum-number-of-steps}}

\begin{Shaded}
\begin{Highlighting}[]
\NormalTok{maxsteps }\OtherTok{\textless{}{-}} \FunctionTok{which.max}\NormalTok{(averagep}\SpecialCharTok{$}\NormalTok{meanSteps)}
\NormalTok{timemax }\OtherTok{\textless{}{-}}  \FunctionTok{gsub}\NormalTok{(}\StringTok{"([0{-}9]\{1,2\})([0{-}9]\{2\})"}\NormalTok{, }\StringTok{"}\SpecialCharTok{\textbackslash{}\textbackslash{}}\StringTok{1:}\SpecialCharTok{\textbackslash{}\textbackslash{}}\StringTok{2"}\NormalTok{, averagep[maxsteps,}\StringTok{\textquotesingle{}interval\textquotesingle{}}\NormalTok{])}
\end{Highlighting}
\end{Shaded}

\begin{itemize}
\tightlist
\item
  Most Steps at: 8:35
\end{itemize}

\begin{center}\rule{0.5\linewidth}{0.5pt}\end{center}

\hypertarget{imputing-missing-values}{%
\subsection{Imputing missing values}\label{imputing-missing-values}}

\hypertarget{calculate-and-report-the-total-number-of-missing-values-in-the-dataset}{%
\paragraph{1. Calculate and report the total number of missing values in
the
dataset}\label{calculate-and-report-the-total-number-of-missing-values-in-the-dataset}}

\begin{Shaded}
\begin{Highlighting}[]
\NormalTok{null }\OtherTok{\textless{}{-}} \FunctionTok{length}\NormalTok{(}\FunctionTok{which}\NormalTok{(}\FunctionTok{is.na}\NormalTok{(data}\SpecialCharTok{$}\NormalTok{steps)))}
\end{Highlighting}
\end{Shaded}

\begin{itemize}
\tightlist
\item
  Number of missing values: 2304
\end{itemize}

\hypertarget{devise-a-strategy-for-filling-in-all-of-the-missing-values-in-the-dataset.}{%
\subparagraph{2. Devise a strategy for filling in all of the missing
values in the
dataset.}\label{devise-a-strategy-for-filling-in-all-of-the-missing-values-in-the-dataset.}}

\hypertarget{create-a-new-dataset-that-is-equal-to-the-original-dataset-but-with-the-missing-data-filled-in.}{%
\subparagraph{3. Create a new dataset that is equal to the original
dataset but with the missing data filled
in.}\label{create-a-new-dataset-that-is-equal-to-the-original-dataset-but-with-the-missing-data-filled-in.}}

\begin{Shaded}
\begin{Highlighting}[]
\NormalTok{DataImputed }\OtherTok{\textless{}{-}}\NormalTok{ data}
\NormalTok{DataImputed}\SpecialCharTok{$}\NormalTok{steps }\OtherTok{\textless{}{-}} \FunctionTok{impute}\NormalTok{(data}\SpecialCharTok{$}\NormalTok{steps, }\AttributeTok{fun=}\NormalTok{mean)}
\end{Highlighting}
\end{Shaded}

\hypertarget{make-a-histogram-of-the-total-number-of-steps-taken-each-day-1}{%
\subparagraph{4. Make a histogram of the total number of steps taken
each
day}\label{make-a-histogram-of-the-total-number-of-steps-taken-each-day-1}}

\begin{Shaded}
\begin{Highlighting}[]
\NormalTok{stepsImputed }\OtherTok{\textless{}{-}} \FunctionTok{tapply}\NormalTok{(DataImputed}\SpecialCharTok{$}\NormalTok{steps, DataImputed}\SpecialCharTok{$}\NormalTok{date, sum)}
\FunctionTok{qplot}\NormalTok{(stepsImputed, }\AttributeTok{xlab=}\StringTok{\textquotesingle{}steps per day (Imputed)\textquotesingle{}}\NormalTok{, }\AttributeTok{ylab=}\StringTok{\textquotesingle{}Frequency\textquotesingle{}}\NormalTok{, }\AttributeTok{binwidth=}\DecValTok{500}\NormalTok{)}
\end{Highlighting}
\end{Shaded}

\includegraphics{Assignment_1_files/figure-latex/unnamed-chunk-11-1.pdf}

\hypertarget{calculate-and-report-the-mean-and-median-total-number-of-steps-taken-per-day.}{%
\subparagraph{Calculate and report the mean and median total number of
steps taken per
day.}\label{calculate-and-report-the-mean-and-median-total-number-of-steps-taken-per-day.}}

\begin{Shaded}
\begin{Highlighting}[]
\NormalTok{MeanImputed }\OtherTok{\textless{}{-}} \FunctionTok{mean}\NormalTok{(stepsImputed)}
\NormalTok{MedianImputed }\OtherTok{\textless{}{-}} \FunctionTok{median}\NormalTok{(stepsImputed)}
\end{Highlighting}
\end{Shaded}

\begin{itemize}
\tightlist
\item
  Mean (Imputed): \ensuremath{1.0766189\times 10^{4}}
\item
  Median (Imputed): \ensuremath{1.0766189\times 10^{4}}
\end{itemize}

\begin{center}\rule{0.5\linewidth}{0.5pt}\end{center}

\hypertarget{are-there-differences-in-activity-patterns-between-weekdays-and-weekends}{%
\subsection{Are there differences in activity patterns between weekdays
and
weekends?}\label{are-there-differences-in-activity-patterns-between-weekdays-and-weekends}}

\hypertarget{create-a-new-factor-variable-in-the-dataset-with-two-levels-weekday-and-weekend-indicating-whether-a-given-date-is-a-weekday-or-weekend-day.}{%
\subparagraph{1. Create a new factor variable in the dataset with two
levels -- ``weekday'' and ``weekend'' indicating whether a given date is
a weekday or weekend
day.}\label{create-a-new-factor-variable-in-the-dataset-with-two-levels-weekday-and-weekend-indicating-whether-a-given-date-is-a-weekday-or-weekend-day.}}

\begin{Shaded}
\begin{Highlighting}[]
\NormalTok{DataImputed}\SpecialCharTok{$}\NormalTok{dateType }\OtherTok{\textless{}{-}}  \FunctionTok{ifelse}\NormalTok{(}\FunctionTok{as.POSIXlt}\NormalTok{(DataImputed}\SpecialCharTok{$}\NormalTok{date)}\SpecialCharTok{$}\NormalTok{wday }\SpecialCharTok{\%in\%} \FunctionTok{c}\NormalTok{(}\DecValTok{0}\NormalTok{,}\DecValTok{6}\NormalTok{), }\StringTok{\textquotesingle{}weekend\textquotesingle{}}\NormalTok{, }\StringTok{\textquotesingle{}weekday\textquotesingle{}}\NormalTok{)}
\end{Highlighting}
\end{Shaded}

\hypertarget{make-a-panel-plot-containing-a-time-series-plot}{%
\subparagraph{2. Make a panel plot containing a time series
plot}\label{make-a-panel-plot-containing-a-time-series-plot}}

\begin{Shaded}
\begin{Highlighting}[]
\NormalTok{avgDataImputed }\OtherTok{\textless{}{-}} \FunctionTok{aggregate}\NormalTok{(steps }\SpecialCharTok{\textasciitilde{}}\NormalTok{ interval }\SpecialCharTok{+}\NormalTok{ dateType, }\AttributeTok{data=}\NormalTok{DataImputed, mean)}
\FunctionTok{ggplot}\NormalTok{(avgDataImputed, }\FunctionTok{aes}\NormalTok{(interval, steps)) }\SpecialCharTok{+} 
    \FunctionTok{geom\_line}\NormalTok{() }\SpecialCharTok{+} 
    \FunctionTok{facet\_grid}\NormalTok{(dateType }\SpecialCharTok{\textasciitilde{}}\NormalTok{ .) }\SpecialCharTok{+}
    \FunctionTok{xlab}\NormalTok{(}\StringTok{"5 minute interval"}\NormalTok{) }\SpecialCharTok{+} 
    \FunctionTok{ylab}\NormalTok{(}\StringTok{"avarage steps"}\NormalTok{)}
\end{Highlighting}
\end{Shaded}

\includegraphics{Assignment_1_files/figure-latex/unnamed-chunk-14-1.pdf}

\end{document}
